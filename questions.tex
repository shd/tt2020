\documentclass[11pt,a4paper,oneside]{article}
\usepackage[utf8]{inputenc}
\usepackage[english,russian]{babel}
\usepackage{amssymb}
%\usepackage{amsmath}
%\usepackage{mathabx}
\usepackage{stmaryrd}
\usepackage[left=2cm,right=2cm,top=2cm,bottom=2cm,bindingoffset=0cm]{geometry}
%\usepackage{bnf}
\newcommand{\lit}[1]{\mbox{`\texttt{#1}'}}
\newcommand{\ntm}[1]{<\mbox{#1}>}
\begin{document}


\begin{center}
\begin{Large}{\bfseries Вопросы к экзамену по курсу <<Теория типов>>}\end{Large}\\
\vspace{1mm}
\begin{small} \itshape ИТМО, группы M3334..M3339\end{small}\\
\small Осенний семестр 2020 г.
\end{center}

\begin{enumerate}
\item Бестиповое лямбда-исчисление. Общие определения (альфа-эквивалентность, бета-редукция, бета-эквивалентность).
Параллельная бета-редукция. Теорема Чёрча-Россера.
\item Нормальный и аппликативный порядок редукций, мемоизация. \textbf{Y}-комбинатор.
Нетипизируемость \textbf{Y}-комбинатора. Слабая и сильная нормализация.
Парадокс Карри, парадокс при интерперетации бестипового лямбда-исчисления как логики.
\item Импликационный фрагмент ИИВ. Теорема о замкнутости И.Ф. относительно доказуемости.
Комбинаторы, базис SKI, его аналог в логике.
\item Просто типизированное лямбда-исчисление. Исчисление по Чёрчу и по Карри. Изоморфизм Карри-Ховарда.
Конъюнкция, дизъюнкция, ложь и соответствующие им конструкции в лямбда-исчислении. Алгебраические типы.
Чёрчевские нумералы. Теорема о выразительной силе просто типизированного лямбда-исчисления (формулировка).
\item Алгебраические термы. Задача унификации в алгебраических термах. Алгоритм унификации. Наиболее общее решение задачи унификации.
\item Задачи проверки типа, реконструкции (вывода) типа, обитаемости типа в просто типизированном лямбда-исчислении.
Их аналоги в интуиционистском исчислении высказываний. 
Алгоритм нахождения типа в просто типизированном лямбда-исчислении. Наиболее общий тип, наиболее общая пара.
\item Логика второго порядка. Выразимость связок через импликацию и квантор всеобщности в интуиционистской логике 
2-го порядка (конъюнкция, дизъюнкция, ложь, отрицание, квантор существования). Простая модель для логики второго порядка.
Система $F$. Изоморфизм Карри-Ховарда для системы $F$: квантор всеобщности, упорядоченные пары, алгебраические типы.
Экзистенциальные типы. Конструкции \texttt{pack} и \texttt{abstype}. Абстрактные типы данных. 
Изоморфизм Карри-Ховарда для квантора существования и экзистенциальных типов.
\item Ранг типа. Частный случай типа. Типы и типовые схемы. Типовая система Хиндли-Милнера. Алгоритм W.
Типизация \textbf{Y}-комбинатора. Экви- и изорекурсивные типы, $\mu$-оператор, \texttt{roll} и \texttt{unroll}.
Примеры конструкций и операторов в языках программирования.
\item Обобщённые типовые системы. Типы, рода, сорта. Лямбда-куб. $\Sigma$ и $\Pi$ типы.
Зависимые типы. Функция \verb!printf!, её типизация.
\item Теория множеств. Необходимость формализации (парадокс Рассела). Равенство. Аксиомы теории множеств 
(равенства, пустого, пары, объединения, степени) и схемы аксиом (выделения, подстановки). Дизъюнктное объединение множеств.
Аксиома выбора и её аналоги (существование частично-обратной к сюръективной функции, существование трансверсали к семейству
непустых множеств). Неконструктивность аксиомы выбора.
\item Частичный, линейный, полный порядок. Порядковые числа (ординалы). Аксиома бесконечности. Существование ординала $\omega$. 
<<Арифметические>> операции на ординалах. Доказательство $\omega+1 \ne 1+\omega$. 
Аналогия между ординалами и упорядоченными алгебраическими типами.
\item Мощность множеств. Кардиналы. Теорема Кантора-Бернштейна. 
\item Диагональный метод и теорема Кантора. Неформальное изложение и изложение на языке аксиом теории множеств.
\item Теорема Лёвенгейма-Сколема. Парадокс Сколема.
\item Интенсиональное и экстенсиональное равенства, достоинства и недостатки подходов.
Равенство как путь в топологическом пространстве. Язык Аренд. Индуктивные типы и равенство. Интервальный тип. 
Стандартные функции: coe, transport, rewrite, path, pmap. 
Функциональная экстенсиональность, её доказуемость в Аренде. $\Sigma$ и $\Pi$ типы в языке Аренд. Примеры использования.
\item Типы, универсумы, пропы, множества. Импредикативность. Сетоиды. Теорема Диаконеску. 
\item Линейная логика. Уникальные типы. Комбинаторный базис $BCKW$.
Полиморфизм (параметрический и наследственный). Система $F_{<:}$ Ядерное и полное правила.
\item Парадокс Бурали-Форте. Парадоксальные универсумы, доказательство существования парадокса Бурали-Форте при 
существовании парадоксального универсума. Общая идея построения парадокса Жирара в системе U.
\end{enumerate}

\end{document}
